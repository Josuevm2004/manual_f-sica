\section{Introducción}

\begin{example}
El análisis del ambiente urbano y los sistemas naturales requiere herramientas que permitan cuantificar fenómenos físicos a diferentes escalas. Este manual aborda dos aplicaciones complementarias de los Sistemas de Información Geográfica (SIG) y la teledetección: el cálculo de energía solar en entornos urbanos y el monitoreo de temperatura superficial en cuerpos de agua.\\
En la primera parte, nos centraremos en el uso de QGIS y el complemento UMEP (Urban Multi-scale Environmental Predictor), específicamente su procesador \textbf{SEBE} (Solar Energy on Building Envelopes), para estimar cuánta energía solar reciben los edificios de una ciudad. Esta herramienta es esencial para arquitectos, urbanistas y científicos interesados en sostenibilidad urbana, energías renovables y diseño bioclimático.\\
En la segunda parte, exploraremos el procesamiento de imágenes satelitales para comprender la evolución de sistemas naturales, tomando como caso de estudio la estimación de la temperatura superficial del Lago Titicaca. Aunque las imágenes satelitales pueden parecer simples fotografías desde el espacio, en realidad representan registros cuantitativos de la radiación emitida o reflejada por la superficie terrestre, captada por sensores remotos que operan en distintas longitudes de onda.\\
Ambas aplicaciones comparten fundamentos físicos comunes relacionados con la radiación electromagnética, pero se especializan en diferentes aspectos del modelado ambiental: uno centrado en el entorno construido y otro en sistemas acuáticos naturales. La integración de estas metodologías permite una visión holística de los procesos ambientales a escala local y regional.\\
\end{example}

\begin{summary}{Recursos Oficiales}{green!60!blue}
Este manual se basa en la documentación oficial de UMEP. Si alguna vez te pierdes o quieres profundizar, consulta siempre la fuente original: 
	\textbf{\url{https://umep-docs.readthedocs.io/en/latest/}}
\end{summary}