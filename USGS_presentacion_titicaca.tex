\subsubsection{ Presentación interpretación y resultados}

\begin{resolution}
Entre los estadísticos descriptivos para el análisis de temperatura superficial del lago Titicaca de 2005 hasta 2025 tenemos a la media, el valor mínimo y valor máximo reunidos en Cuadro 1. 
\end{resolution}


\begin{platform}
\begin{center}
\caption{Estadísticos descriptivos de temperatura del lago Titicaca (2005-2025)}
\begin{tabular}{p{0.2\textwidth} p{0.25\textwidth} p{0.25\textwidth} p{0.25\textwidth}}
\textbf{Año} & \textbf{Media (°C)} & \textbf{Valor Mínimo (°C)} & \textbf{Valor Máximo (°C)} \\
\hline
2005 & 11.854 & 2.908 & 18.47 \\
2006 & 11.329 & 0.003 & 18.494 \\
2007 & 11.586 & 0.003 & 18.751 \\
2008 & 12.229 & 0.003 & 19.376 \\
2009 & 11.808 & 4.169 & 17.896 \\
2010 & 13.356 & 0.003 & 20.484 \\
2011 & 8.447 & 0.003 & 17.718 \\
2012 & 13.071 & 3.181 & 20.367 \\
2013 & 10.990 & 0.003 & 20.118 \\
2014 & 11.053 & 0.003 & 21.071 \\
2015 & 12.877 & 0.003 & 19.243 \\
2016 & 13.250 & 0.003 & 20.74 \\
2017 & 12.753 & 0.003 & 20.19 \\
2018 & 11.996 & 0.003 & 19.116 \\
2019 & 13.507 & 0.003 & 21.557 \\
2020 & 8.911 & 0.003 & 18.566 \\
2021 & 9.475 & 0.003 & 19.004 \\
2022 & 13.623 & 3.328 & 21.092 \\
2023 & 13.363 & 4.2 & 22.815 \\
2024 & 12.201 & 0.003 & 20.76 \\
2025 & 12.288 & 0.003 & 19.345 \\
\end{tabular}
\end{center}
\end{platform}

\begin{example}
Una vez que las imágenes satelitales han sido procesadas para obtener el Índice de Temperatura Superficial (LST), el programa nos brinda gráficas como Figura 2.54, donde claramente se aprecia el rango de temperatura superficial promedio del Lago Titicaca en 2005. Las zonas marcadas con azul representan puntos donde el lago mostró temperaturas frías entre 6 y 10 ºC. 

\end{example}

\begin{figure}[H]
\centering
\begin{imagenbox}
    \centering
    \includegraphics[width=0.85\linewidth]{FOTOS/LST_2005.png}
    \captionof{figure}{Temperatura superficial promedio del Lago Titicaca en 2005 (Fuente: Elaboración propia).}
\end{imagenbox}
\end{figure}

\begin{example}
En contraste con el caso anterior, en Figura 2.55, para 2011, se puede apreciar que el lago presentó un temperatura anual promedio principalmente fría.  
\end{example}
  
\begin{figure}[H]
\centering
\begin{imagenbox}
    \centering
    \includegraphics[width=0.85\linewidth]{FOTOS/LST_2011.png}
    \captionof{figure}{Temperatura superficial promedio del Lago Titicaca en 2011 (Fuente: Elaboración propia).}
\end{imagenbox}
\end{figure}

\begin{example}
En Figura 2.56, podemos apreciar la variación con respecto a la temperatura promedio anual del lago Titicaca entre los años 2005 y 2025. La información que nos ofrece este gráfico nos indica que en los años 2011, 2020 y 2021 el lago registró temperaturas inferiores a los 9 ºC. Un análisis más a fondo para determinar las causas del descenso de la temperatura del lago en dichos años implicaría realizar un análisis durante intervalos de tiempo menores (semestral, trimestral, etc). En ese caso, podría emplearse los datos de imágenes generadas por el satélite MODIS, el cual captura imágenes de manera diaria y, por ende, permite observar tendencias semana a semana. 
\end{example}

\begin{figure}[H]
    \centering
    \includegraphics[width=0.9\textwidth]{FOTOS/PROMEDIOFINAL.png}
    \caption{Variación de la temperatura promedio del lago Titicaca (2005 - 2025) (Fuente: Elaboración propia).}
\end{figure}   

\begin{example}
Con la información obtenida, y haciendo una comparación con estudios recientes en torno a la temperatura anual promedio del Lago Titicaca realizados por otros autores (Han Xu et al., 2021) y SENAMHI (2021), se pudo corroborar la influencia de múltiples factores sobre dicho fenómeno, siendo la radiación del Sol el principal de ellos, dado que registra una intensidad mayor durante el mes de noviembre, en comparación con la temporada de invierno seco (junio-julio) con niveles altos niveles de nubosidad. El contenido de vapor de agua y la altura de la superficie del lago son otros de los factores que hay que tomar en consideración durante el análisis de temperatura superficial del lago Titicaca, ya que, para el caso de las zonas altas, esta tiende a disminuir a un valor mínimo de -6 ºC. 
\end{example}