\section{Metodologías de Descarga de Datos Satelitales}
\section{Introducción a las plataformas de acceso}
En este capítulo se detallan los procedimientos para acceder a las principales plataformas de datos satelitales, con especial énfasis en Google Earth Engine (GEE) debido a su capacidad de procesamiento en la nube y acceso a extensos catálogos de imágenes.

\section{Acceso de cuenta en Google Earth Engine}

\subsection{Requisitos previos}
\begin{example}
    Cuenta de Google válida\\
    Acceso a internet estable\\
    Navegador web actualizado (Chrome, Firefox, Edge)
\end{example}

\subsection{Proceso de registro en Google Cloud}

\subsubsection{Paso 1: Acceso a Google Cloud Platform}
\begin{example}
    Abrir el navegador y dirigirse a: \\
    \texttt{https://console.cloud.google.com/}\\
    Iniciar sesión con la cuenta de Google que se utilizará para GEE\\
    Completar el proceso de verificación de cuenta si es requerido
\end{example}

\begin{figure}[H]
\centering
\begin{imagenbox}
    \centering
    \includegraphics[width=0.85\linewidth]{resources/menu.png}
    \captionof{figure}{Pantalla de acceso a Google Cloud Console}
\end{imagenbox}
\end{figure}

\subsection{Creación de proyecto en Google Cloud}

\subsubsection{Paso 2: Crear nuevo proyecto}
En la parte superior izquierda, hacer clic en el menú desplegable de proyectos.

\begin{figure}[H]
\centering
\begin{imagenbox}
    \centering
    \includegraphics[width=0.85\linewidth]{resources/paso2.0.png}
    \captionof{figure}{Selector de proyectos en Google Cloud Console.}
\end{imagenbox}
\end{figure}

    \begin{example}
    Completar los campos requeridos: 
    \begin{itemize}
        \item \textbf{Nombre del proyecto}: Ejemplo: \textit{Proyecto-GEE-Teledeteccion}.
        \item \textbf{Organización}: Opcional, puede dejarse sin seleccionar.
        \item \textbf{Ubicación}: Predeterminada o según preferencia.
    \end{itemize}
    Hacer clic en \textbf{Crear} y esperar la confirmación. 
    \end{example}
    
\begin{figure}[H]
\centering
\begin{imagenbox}
    \centering
    \includegraphics[width=0.85\linewidth]{resources/confirmacion.png}
    \captionof{figure}{Confirmación de creación de un nuevo proyecto en Google Cloud.}
\end{imagenbox}
\end{figure}


\subsection{Activación de API de Google Earth Engine}

\subsubsection{Paso 4: Habilitar Earth Engine API}
\begin{example}
    Con el proyecto recién creado seleccionado \\
    Dirigirse a: \\ \texttt{https://console.cloud.google.com/apis/library/earthengine.googleapis.com} \\
    Hacer clic en el botón "Habilitar" \\ 
    Esperar la activación (puede tomar algunos segundos) \\
    Verificar que el estado muestre: API habilitada
\end{example}

\begin{figure}[H]
\centering
\begin{imagenbox}
    \centering
    \includegraphics[width=0.85\linewidth]{resources/API.png}
    \captionof{figure}{Activación de Earth Engine API}
\end{imagenbox}
\end{figure}

\subsection{Registro final en Earth Engine}

\subsubsection{Paso 5: Completar registro en GEE}

\begin{example}
    Dirigirse a: \texttt{https://earthengine.google.com/} \\
    Hacer clic en "Get Started" \\
    Seleccionar ''Comenzar'' en el apartado ''Consulta si cumples con los requisitos para el uso no comercial'' \\
    Completar el formulario de registro:
    \begin{itemize}
        \item \textbf{Selecciona el tipo de organización}
        \item \textbf{Verifica la elegibilidad para uso no comercial}
        \item \textbf{Elige tu plan}
        \item \textbf{Describe tu trabajo}
        \item \textbf{Review summary}
    \end{itemize}
    Por último, le das a ''Registrarse''
\end{example}

\section{Verificación de acceso}

\subsection{Confirmación de cuenta activa}
\begin{example}
    Una vez aprobada la cuenta, acceder a: \texttt{https://code.earthengine.google.com/} \\
    Eliges la opción ''I'M AUTHORIZED FOR AN EXISTING CLOUD PROJECT''\\
    Seleccionas el proyecto creado anteriormente ''proyecto-gee-teledeteccion'' y presionas ''SELECT''
\end{example}

\begin{summary}{Resumen del capítulo}{purple!40!blue}
Este capítulo detalla la metodología paso a paso para la obtención de datos satelitales a través de la plataforma \textbf{Google Earth Engine (GEE)}. Se cubren los requisitos previos y el proceso completo de registro, que incluye la creación de un proyecto en \textbf{Google Cloud Platform (GCP)}, la activación de la \textbf{Earth Engine API}, y la finalización del formulario de registro en GEE para uso no comercial. El objetivo es asegurar una cuenta funcional para el posterior procesamiento y análisis de imágenes a gran escala.
\end{summary}