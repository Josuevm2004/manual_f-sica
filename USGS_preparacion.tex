\section{Preparación del Entorno de Trabajo}
\subsection{Instalación de QGIS}
\begin{resolution}
Si aún no tienes instalado el QGIS en tu ordenador, descargalo desde su página oficial: \textbf{\url{https://qgis.org/download/}}\\
Es recomendable la version "Long Term Release (LTR)" por su estabilidad  
\end{resolution}

 \begin{figure}[H]
\centering
\begin{imagenbox}
    \centering
    \includegraphics[width=0.85\linewidth]{Imágenes/LTR.png}
    \captionof{figure}{Página de descarga del QGIS}
\end{imagenbox}
\end{figure}

\subsubsection{Ejecución del instalador}
\begin{resolution}
Al abrir el instalador se nos mostrara el menú de inicio. Le daremos a "Next".
\end{resolution}

\begin{figure}[H]
\centering
\begin{imagenbox}
    \centering
    \includegraphics[width=0.85\linewidth]{Imágenes/P1.png}
    \captionof{figure}{Terminos y condiciones de QGIS}
\end{imagenbox}
\end{figure}


\begin{resolution}
Luego tendremos que aceptar los terminos y condiciones del programa QGIS y le damos a "Next".
\end{resolution}

\begin{figure}[H]
\centering
\begin{imagenbox}
    \centering
    \includegraphics[width=0.85\linewidth]{Imágenes/P2.png}
    \captionof{figure}{Terminos y condiciones de QGIS}
\end{imagenbox}
\end{figure}

\begin{resolution}
Después nos pedirá que definamos la ruta donde se instalará el software y si queremos crear un acceso directo en el escritorio. En nuestro caso dejaremos las opciones por defecto. Le damos nuevamente a "Next".
\end{resolution}


\begin{figure}[H]
\centering
\begin{imagenbox}
    \centering
    \includegraphics[width=0.85\linewidth]{Imágenes/P3.png}
    \captionof{figure}{Ruta de instalación}
\end{imagenbox}
\end{figure}

\begin{resolution}
Ahora le daremos a "Install" para que se empiece a instalar el QGIS en nuestro ordenador
\end{resolution}

\begin{figure}[H]
\centering
\begin{imagenbox}
    \centering
    \includegraphics[width=0.85\linewidth]{Imágenes/P4.png}
    \captionof{figure}{Inicio de la instalación}
\end{imagenbox}
\end{figure}

\begin{resolution}
Una vez haya terminado de instalar se nos mostrará la siguiente ventana. Le daremos a "Finish" para concluir con el proceso de instalación
\end{resolution}

\begin{figure}[H]
\centering
\begin{imagenbox}
    \centering
    \includegraphics[width=0.85\linewidth]{Imágenes/P5.png}
    \captionof{figure}{Fin de la instalación}
\end{imagenbox}
\end{figure}

\subsection{Instalación de UMEP}
Para empezar a utilizar la herramienta SEBE se indicará los pasos detallados para la instalación del complemento \textbf{UMEP} (Urban Multi-scale Environmental Predictor) y su componente de procesamiento en el software de Sistema de Información Geográfica (SIG) \textbf{QGIS}, siguiendo la secuencia de imágenes y texto proporcionados.

\begin{enumerate}
\begin{example}
\item Para la instalación del complemento UMEP, nos dirigimos al botón \textbf{‘Plugins’} o \textbf{‘Complementos’} en la barra superior.
\end{example}

\begin{figure}[H]
\centering
\begin{imagenbox}
    \centering
    \includegraphics[width=0.85\linewidth]{InstalacionUMEP/imagen1.jpeg}
    \captionof{figure}{Paso I}
\end{imagenbox}
\end{figure}

\begin{example}
\item Hacemos clic y seleccionamos la opción \textbf{‘Manage and Install Plugins’} o \textbf{‘Gestionar e instalar complementos’}.
\end{example}

\begin{figure}[H]
\centering
\begin{imagenbox}
    \centering
    \includegraphics[width=0.85\linewidth]{InstalacionUMEP/imagen2.jpeg}
    \captionof{figure}{Paso II}
\end{imagenbox}
\end{figure}
\end{enumerate}

\subsection{Carga de Repositorios}
\begin{enumerate}
\setcounter{enumi}{2} % Continuar la numeración
\begin{example}
\item Se abrirá una ventana emergente. Aquí solo debemos esperar a que se carguen los complementos disponibles del software desde el repositorio oficial de QGIS.
\end{example}

\end{enumerate}

\subsection{Búsqueda e Instalación de UMEP}
\begin{enumerate}
\setcounter{enumi}{3}
\begin{example}
\item Se abrirá una nueva ventana emergente, la ventana principal de \textbf{Plugins (All)}. Nos dirigimos a la barra de búsqueda que se encuentra en la parte de arriba.
\end{example}


\begin{figure}[H]
\centering
\begin{imagenbox}
    \centering
    \includegraphics[width=0.85\linewidth]{InstalacionUMEP/imagen3.jpeg}
    \captionof{figure}{Paso III}
\end{imagenbox}
\end{figure}

\begin{example}
\item Hacemos clic en el buscador y escribimos \textbf{“umep”}.  Aparecerán dos complementos disponibles: \textbf{UMEP} y \textbf{UMEP for processing}. 
\item Seleccionamos la primera opción (\textbf{UMEP}), que es el \textbf{Urban Multi-scale Environmental Predictor}.
\item Hacemos clic en \textbf{‘Install Plugin’}. El complemento se descargará automáticamente.
\end{example}

\begin{figure}[H]
\centering
\begin{imagenbox}
    \centering
    \includegraphics[width=0.85\linewidth]{InstalacionUMEP/imagen4.jpeg}
    \captionof{figure}{Paso IV}
\end{imagenbox}
\end{figure}

\begin{example}
\item En la parte superior de la ventana saldrá un mensaje indicando que la instalación ha sido completada exitosamente (\textit{Plugin installed successfully}).
\item Repetimos este proceso con el segundo complemento: \textbf{UMEP for processing}. 
\end{example}

\newpage
\subsection{Finalización}
\begin{enumerate}
\setcounter{enumi}{9}
\begin{example}
\item Por último, hacemos clic en \textbf{‘Close’} para terminar con la instalación de los complementos y cerrar la ventana.
\end{example}

\begin{figure}[H]
\centering
\begin{imagenbox}
    \centering
    \includegraphics[width=0.85\linewidth]{InstalacionUMEP/imagen5.jpeg}
    \captionof{figure}{Paso V}
\end{imagenbox}
\end{figure}
\end{enumerate}