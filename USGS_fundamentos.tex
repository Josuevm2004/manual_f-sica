
\section{Fundamentos Teóricos}

\begin{tcolorbox}[colback=gray!10, colframe=gray!50!black, title={Principios Comunes de Teledetección y Radiación}]
Tanto el modelado solar urbano como el análisis de temperatura superficial se fundamentan en los principios de la teledetección, definida como la ciencia de obtener información sobre objetos o áreas sin contacto físico directo, mediante el análisis de datos adquiridos por sensores remotos \parencite{lillesand2015}.

Todos los objetos con temperatura superior al cero absoluto ($0$ K) emiten radiación electromagnética. Los sensores capturan esta energía en bandas específicas del espectro:

\begin{itemize}
    \item \textbf{Visible (VIS):} $0.4 - 0.7 \, \mu m$, para características visuales
    \item \textbf{Infrarrojo Cercano (NIR):} $0.7 - 1.3 \, \mu m$, para discriminación de vegetación y agua
    \item \textbf{Infrarrojo Térmico (TIR):} $3.0 - 14 \, \mu m$, para estimación de temperatura superficial
\end{itemize}
\end{tcolorbox}

\begin{tcolorbox}[colback=blue!5, colframe=blue!60!black, title={Fundamentos del Modelado Solar Urbano (SEBE)}]
El plugin \textbf{SEBE} calcula la energía solar potencial píxel a píxel utilizando Modelos Digitales de Superficie (DSM) del terreno y edificios, estimando irradiancia en techos y paredes, con opción de incluir vegetación \parencite{lindberg2018}.
\end{tcolorbox}

\begin{tcolorbox}[colback=green!8, colframe=green!50!black, title={Datos de Entrada Clave}]
\begin{itemize}
    \item \textbf{DSM (Digital Surface Model):} Imagen donde cada píxel representa altura sobre el nivel del mar (incluye terreno y edificios)
    \item \textbf{CDSM (Canopy Digital Surface Model):} Similar al DSM pero para vegetación
    \item \textbf{Datos Meteorológicos:} Radiación global de onda corta, preferiblemente con componentes directos y difusos
\end{itemize}

Según las mejores prácticas en teledetección, estas imágenes requieren buena resolución espacial (idealmente 1 metro o menos para entornos urbanos) \parencite{lillesand2015}.
\end{tcolorbox}

\begin{tcolorbox}[colback=yellow!8, colframe=yellow!50!black, title={Base Física del Cálculo}]
La irradiancia total para un píxel de techo ($R$) se calcula sumando radiación directa, difusa y reflejada mediante la ecuación de Lindberg et al. (2015):

\begin{equation}
    R = \sum_{i=0}^{p} \left[ ( I \omega S + D S + G ( 1 - S ) \alpha ) \right]
    \label{eq:radiacion_total}
\end{equation}

Donde:
\begin{itemize}
    \item $p$: Número de parches en el hemisferio
    \item $I$: Radiación directa incidente
    \item $D$: Radiación difusa
    \item $G$: Radiación global del i-ésimo parche
    \item $\alpha$: Albedo de la superficie
    \item $S$: Sombra calculada para cada píxel
    \item $\omega$: Ángulo de incidencia del sol
\end{itemize}

El ángulo de incidencia del sol ($\omega$) se define como:

\begin{equation}
    \omega = \sin(\text{pendiente}) \cos(\text{alt}) \cos(\text{az} - \text{aspecto}) + 
    \cos(\text{pendiente}) \sin(\text{alt})
\end{equation}
\end{tcolorbox}


\subsubsection{Datos de Entrada y Salida para SEBE}

\begin{concept}
\begin{center}
\begin{tabular}{p{0.35\textwidth} p{0.6\textwidth}}
\textbf{Tipo de aplicación} & \textbf{Utilidad en investigación} \\
\hline
Visualización preliminar & Reconocimiento rápido del área de estudio \\
Análisis temporal & Comparación de cambios a través del tiempo \\
Georreferenciación & Establecer puntos de control en terreno \\
Planificación de campo & Identificar sitios de muestreo óptimos \\
Comunicación & Presentar resultados de forma visual \\
\hline
\end{tabular}
\end{center}
\end{concept}


\subsection{Fundamentos del Análisis de Temperatura Superficial}

\subsubsection{El Programa Landsat}

Para estudios de series temporales como el del Lago Titicaca, se utilizan datos de misiones Landsat \parencite{roy2014}:


\begin{concept}
\begin{center}
\begin{tabular}{p{0.4\textwidth}| p{0.55\textwidth}}
    \textbf{Landsat 5 (1984-2011)}& Sensor TM, banda térmica a 120 m nativa\\ \hline
    \textbf{Landsat 7 (1999-presente)}& Sensor ETM+, resolución térmica de 60 m\\ \hline
    \textbf{Landsat 8 (2013-presente)}& Sensores OLI y TIRS, dos bandas térmicas a 100 m nativa\\ \hline
    \textbf{Landsat 9 (2021-presente)}& Continuación de Landsat 8 con mejoras en calibración
\end{tabular}
\end{center}    
\end{concept}


Todas tienen resolución temporal de 16 días y órbita polar heliosincrónica, pasando sobre el ecuador aproximadamente a las 10:00 a.m. hora local.

\subsubsection{Física de la Temperatura Superficial Terrestre (LST)}

La LST es la temperatura radiativa de la "piel" superficial, diferente de la temperatura del aire. Su estimación se fundamenta en la Ley de Stefan-Boltzmann \parencite{sobrino2004}:


\begin{summary}{Ley de Stefan-Boltzmann}{cyan!50!blue}
\begin{equation}
    M = \sigma T^4
    \label{eq:stefan_boltzmann}
\end{equation}
Donde $M$ es la exitancia radiante espectral, $\sigma$ es la constante de Stefan-Boltzmann ($5.67 \times 10^{-8} \, \text{W m}^{-2} \text{K}^{-4}$) y $T$ es la temperatura absoluta. \\
Dado que los objetos naturales no son cuerpos negros perfectos, se considera la \textbf{emisividad} ($\varepsilon$), relación entre la energía radiada por un material y la que radiaría un cuerpo negro a la misma temperatura. Para el agua, $\varepsilon \approx 0.98 - 0.99$, facilitando estimaciones precisas.
\end{summary}



\subsection{Plataformas Computacionales}

\subsubsection{Procesamiento Local (QGIS/UMEP)}

Para modelado solar urbano, se utiliza QGIS con el complemento UMEP, permitiendo análisis detallado a escala de edificio con control completo sobre parámetros de entrada.

\subsubsection{Procesamiento en la Nube (Google Earth Engine)}

Para análisis de grandes volúmenes de datos satelitales históricos, Google Earth Engine (GEE) ofrece ventajas significativas \parencite{gorelick2017}:

\begin{example}
    Almacena petabytes de imágenes (catálogo completo Landsat y Sentinel)
    \\Permite ejecución de algoritmos complejos sin descarga local
    \\Arquitectura de procesamiento paralelo para análisis rápidos
    \\Facilita validación con técnicas avanzadas
\end{example}

Estudios recientes demuestran que GEE permite generar mapas de LST a 10m con RMSE de aproximadamente 3.6K \parencite{garcia-santos2025}, y combinaciones Landsat-Sentinel para patrones térmicos urbanos detallados \parencite{onacillova2022}.

Ambas aproximaciones demuestran cómo la teledetección y el modelado espacial transforman datos brutos en información valiosa para la toma de decisiones ambientales.

\subsection{Diagrama del Proceso SEBE}
	
	\begin{figure}[H]
		\centering
		\fbox{\includegraphics[width=1\linewidth]{Imágenes/Fundamentos/Flujo_SEBE.png}}
		\caption{Flujo de trabajo simplificado del modelo SEBE}
		\label{fig:flujosebe}
	\end{figure}

\subsection{Ejemplo de Resultado}

 A continuación se muestra un ejemplo conceptual de la salida gráfica del modelo (mapa de irradiancia):
	
	\begin{figure}[H]
		\centering
		\fbox{\includegraphics[width=0.6\linewidth]{Imágenes/Fundamentos/MapaDelPeruIrradiacionEnergiaSolar.png}}
		\caption{Mapa del Perú de Irradiacion de energía solar promedio}
		\label{fig:mapadelperuirradiacionenergiasolar}
	\end{figure}
 