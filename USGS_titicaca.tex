\subsection{Adquisición y Procesamiento de Datos Satelitales: Caso Lago Titicaca}
\subsubsection{Definir la ubicación de referencia:}
\begin{example}
\begin{enumerate}
\item Escribir el lugar de referencia en la barra de búsqueda.
\item Hacer clic en \textit{Show}.
\item Seleccionar la opción que se muestra (si no aparece, escribir el país).
\item Verificar que aparezcan las coordenadas y un pin en la ubicación.
\end{enumerate}
\end{example}

\begin{figure}[H]
\centering
\begin{imagenbox}
    \centering
    \includegraphics[width=0.85\linewidth]{DESCARGAS/D2.jpg}
    \captionof{figure}{Proceso para delimitar la región de interés.}
\end{imagenbox}
\end{figure}

\subsubsection{Seleccionar fechas de búsqueda}
\begin{example}
\begin{enumerate}
    \item Seleccionar fecha de inicio del periodo de búsqueda.
    \item Seleccionar fecha final del periodo.
    \item Hacer clic en \textit{Data Sets}.
\end{enumerate}
\end{example}

\begin{figure}[H]
\centering
\begin{imagenbox}
    \centering
    \includegraphics[width=0.85\linewidth]{DESCARGAS/D3.jpg}
    \captionof{figure}{Proceso para seleccionar el periodo de búsqueda.}
\end{imagenbox}
\end{figure}

\subsubsection{Selección de Data Sets}
\begin{resolution}
Aquí se deben seleccionar los satélites y tipos de imágenes que se desean descargar.  
En Earth Explorer hay una gran cantidad de sensores disponibles; a continuación se muestran los pasos para seleccionar Landsat 8 y MODIS (LST y NDVI).
\end{resolution}

\begin{example}
\textbf{Seleccionar imágenes Landsat 8}
\begin{enumerate}
\item Navegar a:
\begin{itemize}
\item \texttt{Landsat > Landsat Collection 1 Level-1 > Landsat 8 OLI/TIRS C1 Level-1}
\end{itemize}
\end{enumerate}
\end{example}

\begin{figure}[H]
\centering
\begin{imagenbox}
    \centering
    \includegraphics[width=0.85\linewidth]{DESCARGAS/D4.jpg}
    \captionof{figure}{Proceso para seleccionar datos de LST de Landsat.}
\end{imagenbox}
\end{figure}

\begin{example}
\textbf{Seleccionar imágenes LST de MODIS}
\begin{enumerate}
\item Navegar a:
\begin{itemize}
\item \texttt{NASA LPDAAC Collections > MODIS Land Surface Temp and Emiss - V6}
\item \texttt{MODIS MOD11A1 V6} (LST día y noche, 1 km-pixel, diario)
\end{itemize}
\end{enumerate}
\end{example}

\begin{figure}[H]
\centering
\begin{imagenbox}
    \centering
    \includegraphics[width=0.85\linewidth]{DESCARGAS/D5.jpg}
    \captionof{figure}{Proceso para seleccionar datos de LST de MODIS.}
\end{imagenbox}
\end{figure}

\begin{example}
\textbf{Seleccionar imágenes NDVI de MODIS}

\begin{enumerate}
    \item Navegar a:
    \begin{itemize}
        \item \texttt{NASA LPDAAC Collections > MODIS Vegetation Indices – V6}
        \item \texttt{MODIS MOD13Q1 V6} (NDVI y EVI, 250 m-pixel, periodo 16 días)
    \end{itemize}
\end{enumerate}
\end{example}

\begin{figure}[H]
\centering
\begin{imagenbox}
    \centering
    \includegraphics[width=0.85\linewidth]{DESCARGAS/D6.jpg}
    \captionof{figure}{Proceso para seleccionar datos de NDVI de MODIS.}
\end{imagenbox}
\end{figure}

\begin{example}
\textbf{Ir a resultados}

\begin{enumerate}
    \item Hacer clic en \textit{Results}.
    \item Seleccionar el \textit{Data Set} del que se desean ver los resultados.
\end{enumerate}
\end{example}

\begin{figure}[H]
\centering
\begin{imagenbox}
    \centering
    \includegraphics[width=0.85\linewidth]{DESCARGAS/D7.jpg}
    \captionof{figure}{Proceso para mostrar el resultado de la búsqueda.}
\end{imagenbox}
\end{figure}

\begin{example}
Los resultados se muestran en páginas de 10 imágenes ordenadas de la más reciente a la más antigua.
\end{example}

\begin{figure}[H]
\centering
\begin{imagenbox}
    \centering
    \includegraphics[width=0.85\linewidth]{DESCARGAS/D8.jpg}
    \captionof{figure}{Proceso para seleccionar entre las opciones de datos.}
\end{imagenbox}
\end{figure}
            
\subsubsection{Seleccionar imagen útil de Landsat}

\begin{example}
    \begin{enumerate}
    \item Visualizar la imagen sobre el área de estudio.
    \end{enumerate}
\end{example}

\begin{figure}[H]
\centering
\begin{imagenbox}
    \centering
    \includegraphics[width=0.85\linewidth]{DESCARGAS/D9.jpg}
    \captionof{figure}{Proceso para visualizar los datos encontrados.}
\end{imagenbox}
\end{figure}

\begin{example}
 \begin{enumerate}
    \item Cambiar la imagen base a \textit{Map} para facilitar la visualización.
    \item Seleccionar una fecha con \textbf{cielo despejado} en el área de estudio.
    \item Hacer clic en \textit{Download}.
    \end{enumerate}
\end{example}
   
\begin{figure}[H]
\centering
\begin{imagenbox}
    \centering
    \includegraphics[width=0.85\linewidth]{DESCARGAS/D10.jpg}
    \captionof{figure}{Proceso para descargar los datos seleccionados.}
\end{imagenbox}
\end{figure}

\subsubsection{Descargar imagen de Landsat}

\begin{example}
\begin{enumerate}
    \item Si se solicita, iniciar sesión con el usuario USGS.
    \item Al iniciar sesión nuevamente, hacer clic en el ícono de descarga de la imagen Landsat.
    \item Descargar el producto \textbf{Level-1 GeoTIFF Data Product}.
\end{enumerate}
\end{example}